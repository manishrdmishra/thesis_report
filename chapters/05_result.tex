	% !TeX root = ../main.tex
	% Add the above to each chapter to make compiling the PDF easier in some editors.
\chapter{Result}\label{chapter:result}


%\section{Results and Discussion}
\section{Evaluation of ClassUSK, SegUSK and CombUSK}
We evaluate ClassUSK, SegUSK and our combination framework (termed
CombUSK here) on our reserved test dataset (30 images, 15 healthy tissue, 15 tumour). We first consider the classification accuracy on pixel level. As shown in Fig.
3a, both ClassUSK and CombUSK gain a high pixel classification accuracy of
$87.8 \pm l4.2 \%$ (mean and s.d.) and $88.1 \pm 4.2 \%$ respectively. By contrast, the performance of
SegUSK (mitochondria vs non-mitochondrial structures) is poorer($80.3 \pm 5.7 \%$, as it
tends to predict impurities in the background into the foreground, i.e. mitochondrial
class. The evaluation criteria for object segmentation is based on counting the number
of correctly segmented mitochondria (Dice coefficient > 80 \%), divided by the
average of total number of ground-truth mitochondria and that of automatic
segmented mitochondria (similar to the definition of the Dice coefficient, but number
based rather than pixel based). For ClassUSK, mitochondrial segmentation is
generated through labeling the connected components of the inversed label map of
the predicted non-mitochondrial class. This approach performs poorly when
mitochondria are nearby and touching, since ClassUSK tends to merge them into a
single mitochondrion (shown in the exemplary segmentation of Fig. 3b, red arrows).
In comparison, although SegUSK performs better in separating nearby mitochondria
and generating more clear boundaries, it cannot differentiate impurities from real
mitochondria probably since both have similar shape features. The best performance
an astonishing accuracy of $98.2 \pm 2.8 \%$ (around half a mitochondrion error per image),
is achieved with CombUSK, as it effectively overcomes the disadvantages of both
networks

\section{Heterogeneity quantification of mitochondrial structures}
Our score allows us to assign single mitochondria to healthy and cancerous states, but
also to states in between the two extremes. In Fig. 4a, we show exemplary images of
mitochondria for different ranges of our score. The structural differences suggest that
the transition from healthy to cancerous mitochondria is not an instant switching, but
rather a continuous process. This is quantified by our structural score: the intensely
stained mitochondrial matrix tends to connect the outer-membrane on both sides in
mitochondria with a low score. These connections begin to break up when the score
increases and the matrix shrinks into isolated dots in mitochondria with a high score.
Distributions of scores for different samples exhibit the clear differences between
healthy and cancerous cells: the majority of healthy mitochondria have a score under
0.2 (Fig. 4b), while most of cancerous mitochondria have a score above 0.8. An
exception is the HepT1 tumour cell line, which shows an increased heterogeneity in
structural scores as compared to other tumour cell lines. This is an inspiring finding
as HepT1 is known to be resistant to chemotherapy. The varying mitochondrial
structural patterns identified by our structural score might reflect the resistance
against single chemotherapeutics.

\begin{figure}[ht!]
\includegraphics[scale=.7]{./figures/figure_3.png}
\caption{CombUSK gives rise to highly accurate mitochondria segmentation. (a) Evaluation of ClassUSK, SegUSK and CombUSK on 30 test images. (b) Exemplary mitochondrial segmentation (ground-truth: white outlines, red: ClassUSK, green: SegUSK, blue: CombUSK, note that Cytan outlines are due to an overlay effect of SegUSK and CombUSK). Red arrow indicates a merged segmentation of three closely-attached mitochondria by ClassUSK. Green arrow indicates an miss-segmentation of impurities as mitochondria by SegUSK.}
\end{figure}
\begin{figure}[h!]
\includegraphics[scale=0.7]{./figures/figure_4.png}
\caption{Our score visualizes the transition between healthy and cancerous mitochondria. (a) A score from 0 to 1 represents the structural transition from healthy to cancerous mitochondria, with six exemplary mitochondria shown on the top.}
\end{figure}
\section{Single mitochondrion score visualization}


\begin{figure}[htb]
    \begin{subfigure}{.5\textwidth}
         \centering
        \includegraphics[width=1\linewidth]{./figures/Fao_seg_color_826-14_200000_11.png}
        \caption{Fao test image}
        \label{fig:fao_raw}
    \end{subfigure}
  \begin{subfigure}{.5\textwidth}
        \centering
       \includegraphics[width=1\linewidth]{./figures/Fao_mitochondrion_score_826-14_200000_11.png}
        \caption{Fao predicted segmentation}
        \label{fig:healthy_raw}
    \end{subfigure}
    
    
     \begin{subfigure}{.5\textwidth}
         \centering
        \includegraphics[width=1\linewidth]{./figures/H4IIE_seg_color_330-12_20000_7.png}
        \caption{H4IIE test image}
        \label{fig:fao_raw}
    \end{subfigure}
  \begin{subfigure}{.5\textwidth}
        \centering
       \includegraphics[width=1\linewidth]{./figures/H4IIE_mitochondrion_score_330-12_20000_7.png}
        \caption{H4IIE predicted segmentation}
        \label{fig:healthy_raw}
    \end{subfigure}
    
    
     \begin{subfigure}{.5\textwidth}
         \centering
        \includegraphics[width=1\linewidth]{./figures/HepG2_seg_color_230-12_20000_7.png}
        \caption{HepG2 test image}
      \label{fig:hepg2_test}
    \end{subfigure}
  \begin{subfigure}{.5\textwidth}
        \centering
       \includegraphics[width=1\linewidth]{./figures/HepG2_mitochondrion_score_230-12_20000_7.png}
        \caption{HepG2 predicted segmentation}
       \end{subfigure}
    \end{figure}
\begin{figure}[htb!]
\ContinuedFloat

\begin{subfigure}{.5\textwidth}
         \centering
        \includegraphics[width=1\linewidth]{./figures/HepT1_seg_color_98-13_20000_2.png}
        \caption{HepT1 test image}
      
    \end{subfigure}
  \begin{subfigure}{.5\textwidth}
        \centering
       \includegraphics[width=1\linewidth]{./figures/HepT1_mitochondrion_score_98-13_20000_2.png}
        \caption{HepT1 predicted segmentation}
 \end{subfigure}

     \begin{subfigure}{.5\textwidth}
        \centering
        \includegraphics[width=1\linewidth]{./figures/Healthy_379-12b_20000_2_seg_color.png}
        \caption{Healthy test image}
        \label{fig:healthy_label}
    \end{subfigure}
    \begin{subfigure}{.5\textwidth}
        \centering
        \includegraphics[width=1\linewidth]{./figures/Healthy_379-12b_20000_2_projecte_score.png}
        \caption{Healthy predicted segmentation}
        \label{fig:healthy_label}
    \end{subfigure}
    
         \begin{subfigure}{.5\textwidth}
        \centering
        \includegraphics[width=1\linewidth]{./figures/McA7777_seg_color_377-12b_20000_3.png}
        \caption{Healthy test image}
        \label{fig:healthy_label}
    \end{subfigure}
    \begin{subfigure}{.5\textwidth}
        \centering
        \includegraphics[width=1\linewidth]{./figures/McA7777_mitochondrion_score_377-12b_20000_2.png}
        \caption{Healthy predicted segmentation}
        \label{fig:healthy_label}
    \end{subfigure}
\end{figure}

\begin{figure}[htb!]
\ContinuedFloat

\begin{subfigure}{.5\textwidth}
        \centering
        \includegraphics[width=1\linewidth]{./figures/Mhic1_seg_color_825-14_200000_13.png}
        \caption{Mhic1 test image}
        \label{fig:healthy_label}
    \end{subfigure}
    \begin{subfigure}{.5\textwidth}
        \centering
        \includegraphics[width=1\linewidth]{./figures/Mhic1_mitochondrion_score_825-14_200000_13.png}
        \caption{Mhic1 predicted segmentation}
        \label{fig:healthy_label}
    \end{subfigure}    

  \caption{Projected scores of individual mitochondrion using ClassUSK and SegUSK}
    \label{fig:mitochondrion_score} 
\end{figure}

\section{Biological replicates wise analysis}
Each cell line have one or more biological replicates. Table \ref{table:replicates} show the list of biological replicates corresponding to cell lines. We wanted to do Anova1 analysis for each cell line. The purpose of one-way ANOVA is to determine whether data from several groups (levels) of a factor have a common mean. That is, one-way ANOVA enables you to find out whether different groups of an independent variable have different effects on the response variable y.

\begin{table}
\begin{center}
\begin{tabular}{ |c|c|c| } 
\hline
Tumor subtype  & Biological replicates \\
\hline
\multirow{3}{4em}{H4IIE} & 330 - 12  \\ 
&L6512 \\ 
& L13412\\ 
& L18912 \& 189-12b \\
\hline
\multirow{3}{4em}{Healthy} & L5612 \& 56-12  \\ 
&331 - 14 \\ 
& 606 - 14\\ 
\hline
\multirow{3}{4em}{McA7777} & T14912 \\
& T15412 \\
& T37113 \\
& T37713 \\
\hline
\end{tabular}
\end{center}
\label{table:replicates}
\caption{Biological replicates}
\end{table}


\begin{figure}[htb!]
    \begin{subfigure}{1\textwidth}
         \centering
        \includegraphics[width=1\linewidth]{./figures/biological_replicates/H4IIE_box_plot.png}
        \caption{Box plot for H4IIE cell line}
   \label{fig:fao_raw}
    \end{subfigure}
    \vspace{4em}
  \begin{subfigure}{1\textwidth}
        \centering
       \includegraphics[width=1\linewidth]{./figures/biological_replicates/H4IIE_means_comp.png}
        \caption{Mean comparison for H4IIE cell line}
        \label{fig:healthy_raw}
    \end{subfigure}
    
    \end{figure}
    \begin{figure}[htb!]
\ContinuedFloat
    \begin{subfigure}{1\textwidth}
         \centering
        \includegraphics[width=1\linewidth]{./figures/biological_replicates/leber_box_plot.png}
        \caption{Box plot for Healthy cell line}
   \label{fig:fao_raw}
    \end{subfigure}
    \begin{subfigure}{1\textwidth}
         \centering
        \includegraphics[width=1\linewidth]{./figures/biological_replicates/leber_means_comp.png}
        \caption{Mean comparison for for Healthy cell line}
   \label{fig:fao_raw}
    \end{subfigure}
    
     \end{figure}
    \begin{figure}[htb!]
\ContinuedFloat

 \begin{subfigure}{1\textwidth}
         \centering
        \includegraphics[width=1\linewidth]{./figures/biological_replicates/tumor_box_plot.png}
        \caption{Box plot for McA7777 cell line}
   \label{fig:fao_raw}
    \end{subfigure}
    \begin{subfigure}{1\textwidth}
         \centering
        \includegraphics[width=1\linewidth]{./figures/biological_replicates/tumor_means_comp.png}
        \caption{Mean comparison for for McA7777 cell line}
   \label{fig:fao_raw}
    \end{subfigure}
    \label{fig:replicate_box_plot}
\caption{Box plot shows the average score of tumor for a biological replicate. The name of replicates are written in the form "x--y", where x is the name of the replicate and y is the number of mitochondrion extracted after segmentation. \\
Mean comparison plot shows how one replicate differs from other in their mean tumor score. X-axis of mean comparison plot show the mean tumor score and Y-axis represents the replicates. }
\end{figure}