	% !TeX root = ../main.tex
	% Add the above to each chapter to make compiling the PDF easier in some editors.
\chapter{Method}\label{chapter:method}


\label{sec:method}
We employ a joint segmentation-classification framework to derive structural score
for each mitochondrion (Fig. 2). For each electron micrograph, a pixel classification
module generates three probability maps for the healthy mitochondrial, cancerous
mitochondrial, and non-mitochondrial classes (Fig. 2d). In parallel, an object
segmentation module identifies each mitochondrion as a region of interest (Fig. 2e).
Both networks use theU-shape sparse kernel structure [ref], and are termed ClassUSK
and SegUSK, respectively, and differ in output layer and the loss function (Fig. 2b
and c). The outputs of both networks are overlaid on the original image: the
transparent green, red and blue represents pixel-level probability of healthy
mitochondrial, cancerous mitochondrial and non-mitochondrial classes, whilst the white outlines are segmented object contours (Fig. 2f). The quantification of individual mitochondrial structure combines the results of the two networks, with details given
in the Subsection 2.3.
The USK-net is a combination of the U-shape subnetwork (U-net) \cite{DBLP:conf/miccai/RonnebergerFB15} and sparse
kernel subnetwork (SK-net) \cite{DBLP:journals/corr/LiZW14}. Like U-net, the USK-net has a contracting path and
an expanding path, yet with only one downsampling/upsampling step in both paths
 as shown in fig\ref{fig:usk-net_arch}, three steps less than the original U-net structure. The bottom of the network
The bottom of the network is organized as the SK-net, where the sparse kernel technique is used 
to eliminate
almost all of the redundant computation among connected patches and to accelerate
the network training and testing \cite{DBLP:journals/corr/LiZW14}. Compare to U-net and SK-net architectures of
similar input and output size, USK-net has only 25$ \%$ of U-net and SK-net weights
and hence is much easier to be trained, particularly with limited amount of data.
\begin{figure}[ht!]
\centering 
\includegraphics[scale=0.7]{./figures/training_detail_fig_2.png}
\caption{Schematic plot of our joint segmentation-classification framework to derive structural score for each intact mitochondrion. See text for explanation. }
\label{fig:schematic_seg_classification}
\end{figure}


\begin{figure}[ht!]
\includegraphics[scale=0.5]{./figures/segUSK_arch.png}
\caption{Architecture of classUSK and SegUSK }
\label{fig:classUSK}
\end{figure}


\section{Pixel classification}
We classify each pixel into three classes using ClassUSK:(i) healthy mitochondrial,
(ii) cancerous mitochondrial and (iii) non-mitochondrial structures. We use a
soft-max layer as our output layer so that we can obtain a continuous score that
measures the probability of one pixel belong to each class, rather than a single label.
In the original USK-net structure proposed in [ref], the context
size is $180 \times 180$  pixels. First we trained our model using original USK-net and tested on test images. We can see that the segmentation obtained from this model in figure \ref{fig:context_180}


\begin{figure}[th!]
    \begin{subfigure}{.5\textwidth}
         \centering
        \includegraphics[width=1\linewidth]{./figures/Fao_826-14_200000_11.png}
        \caption{Fao test image}
        \label{fig:fao_raw}
    \end{subfigure}
  \begin{subfigure}{.5\textwidth}
        \centering
       \includegraphics[width=1\linewidth]{./figures/Fao_seg_826-14_200000_11.png}
        \caption{Fao predicted segmentation}
        \label{fig:healthy_raw}
    \end{subfigure}
    \begin{subfigure}{.5\textwidth}
        \centering
        \includegraphics[width=1\linewidth]{./figures/Healthy_379-12b_20000_2.png}
        \caption{Healthy test image}
        \label{fig:healthy_label}
    \end{subfigure}
    \begin{subfigure}{.5\textwidth}
        \centering
        \includegraphics[width=1\linewidth]{./figures/Healthy_seg_379-12b_20000_2.png}
        \caption{Healthy predicted segmentation}
        \label{fig:healthy_label}
    \end{subfigure}

\begin{subfigure}{.5\textwidth}
        \centering
        \includegraphics[width=1\linewidth]{./figures/Mhic1_825-14_200000_13.png}
        \caption{Mhic1 test image}
        \label{fig:healthy_label}
    \end{subfigure}
    \begin{subfigure}{.5\textwidth}
        \centering
        \includegraphics[width=1\linewidth]{./figures/Mhic1_seg_825-14_200000_13.png}
        \caption{Mhic1 predicted segmentation}
        \label{fig:healthy_label}
    \end{subfigure}    
  
  \caption{Segmentation visualization of predicted images by the model trained with the context of $180 \times 180 $,we can see that some mitochondria are touching each other and some are broken.}  
  \label{fig:context_180}  
\end{figure}


One key factor of a successful classification is that the context size considered for
around each pixel in its classification should be large enough to cover an entire
mitochondrion, as it is difficult to differentiate a fraction of the inner structures of a
intact mitochondrion from impurities that include a broken mitochondrial fragment
(as shown in Fig. \ref{fig:all_mitochondria}).

\begin{figure}[ht!]
\centering
        \includegraphics[width=1\linewidth]{./figures/all_mitochondria.png}
        \caption{Electron micrographs of isolated mitochondria from healthy mouse liver tissue
(green boxes) and different liver tumor cell lines (red boxes, bars equal 1 /mu m). Intra-
class variation is found for both healthy and tumor samples: While some mitochondria
from healthy liver cells resemble cancerous structures (indicated by the red arrow)
there are also mitochondria in cancer samples that resemble the healthy morphology
(indicated by the green arrow in H4IIE). Furthermore, the images are corrupted by
impurities and fragments from mitochondria and other organelles (indicated by blue
arrows).}
        \label{fig:all_mitochondria}
\end{figure}

From this experiment we observed that context size of $180 \times 180 $ is not sufficient for our application, as some
mitochondria in specific cancer cell lines (e.g. Fao) have a very large size. Hence, we use two downsampling and upsampling layers to increase the context size of our USK-net to $300 \times 300$ pixels (Figure \ref{fig:classUSK}). 

\begin{figure}[th!]
    \begin{subfigure}{.5\textwidth}
         \centering
        \includegraphics[width=1\linewidth]{./figures/Fao_826-14_200000_11.png}
        \caption{Fao test image}
        \label{fig:fao_raw}
    \end{subfigure}
  \begin{subfigure}{.5\textwidth}
        \centering
       \includegraphics[width=1\linewidth]{./figures/Fao_seg_large_context_size_826-14_200000_11.png}
        \caption{Fao predicted segmentation}
        \label{fig:healthy_raw}
    \end{subfigure}
    \begin{subfigure}{.5\textwidth}
        \centering
        \includegraphics[width=1\linewidth]{./figures/Healthy_379-12b_20000_2.png}
        \caption{Healthy test image}
        \label{fig:healthy_label}
    \end{subfigure}
    \begin{subfigure}{.5\textwidth}
        \centering
        \includegraphics[width=1\linewidth]{./figures/Healhty_seg_large_context_size_379-12b_20000_2.png}
        \caption{Healthy predicted segmentation}
        \label{fig:healthy_label}
    \end{subfigure}

\begin{subfigure}{.5\textwidth}
        \centering
        \includegraphics[width=1\linewidth]{./figures/Mhic1_825-14_200000_13.png}
        \caption{Mhic1 test image}
        \label{fig:healthy_label}
    \end{subfigure}
    \begin{subfigure}{.5\textwidth}
        \centering
        \includegraphics[width=1\linewidth]{./figures/Mhic1_seg_large_context_size_825-14_200000_13.png}
        \caption{Mhic1 predicted segmentation}
        \label{fig:healthy_label}
    \end{subfigure}    
  
  \caption{Segmentation visualization of predicted images by the model trained with the context of $300 \times 300 $,we can see that segmentation results are better than with the context size of $ 180 \times 180 $.}  
\end{figure}


\section{Object segmentation}
Since our aim is to derive one score per mitochondrion, we have to also segment
individual mitochondria as objects of interest. A simple way to obtain segmentation is
to invert the classification result of the non-mitochondrial structures (class iii) , which we were doing. This
approach however is error-prone in practice, since e.g. as small amount of
miss-classified pixels between two closely neighboring mitochondria could merge
them as a single mitochondrion, and missclassified pixels inside one mitochondrion
would break into two or more individual mitochondria (an evaluation is provided in
the Results section). Since this kind of miss
segmentation is usually difficult to be
restored with post-processing techniques (such as morphological filtering), we train a
separate network to achieve an end-to-end segmentation. \\
The architecture of our SegUSK is very similar to ClassUSK, but with an affinity
layer and a component layer after soft-max layer (Figure \ref{fig:classUSK}). The affinity layer explores
neighboring pixels both in the horizontal and vertical directions and compute
connectivity among pixels. The component layer is a small layer based on the
OpenCV flood-filling algorithm and outputs separated connected components from a
foreground-background labeled ground truth. The Malis (maximum affinity learning
of image segmentation) loss function takes the output of the affinity and component
layers and attributes the error map between the prediction and the ground truth for
back propagation ( refer to \cite{DBLP:journals/corr/Tschopp15} and \cite{DBLP:journals/corr/abs-0911-5372} for details of Affinity layer, component layer
and Malis loss calculation). 
It should be also noted that we only use SegUSK to segment mito-
chondria from background, regardless of its tissue type.

In figure \ref{fig:seg_SegUSK} we can visualize the segmentation obtained by SegUSK network.
\begin{figure}[ht!]
    \begin{subfigure}{.5\textwidth}
         \centering
        \includegraphics[width=1\linewidth]{./figures/Fao_826-14_200000_11.png}
        \caption{Fao test image}
        \label{fig:fao_raw}
    \end{subfigure}
  \begin{subfigure}{.5\textwidth}
        \centering
       \includegraphics[width=1\linewidth]{./figures/Fao_seg_net2_826-14_200000_11.png}
        \caption{Fao predicted segmentation}
        \label{fig:healthy_raw}
    \end{subfigure}
    
    
     \begin{subfigure}{.5\textwidth}
         \centering
        \includegraphics[width=1\linewidth]{./figures/H4IIE_330-12_20000_7.png}
        \caption{H4IIE test image}
        \label{fig:fao_raw}
    \end{subfigure}
  \begin{subfigure}{.5\textwidth}
        \centering
       \includegraphics[width=1\linewidth]{./figures/H4IIE_seg_net2_330-12_20000_7.png}
        \caption{H4IIE predicted segmentation}
        \label{fig:healthy_raw}
    \end{subfigure}
    
    
     \begin{subfigure}{.5\textwidth}
         \centering
        \includegraphics[width=1\linewidth]{./figures/HepG2_230-12_20000_7.png}
        \caption{HepG2 test image}
      \label{fig:hepg2_test}
    \end{subfigure}
  \begin{subfigure}{.5\textwidth}
        \centering
       \includegraphics[width=1\linewidth]{./figures/HepG2_seg_net2_230-12_20000_7.png}
        \caption{HepG2 predicted segmentation}
       \end{subfigure}
    \end{figure}
\begin{figure}[ht!]
\ContinuedFloat

\begin{subfigure}{.5\textwidth}
         \centering
        \includegraphics[width=1\linewidth]{./figures/HepT1_98-13_20000_3_HepT1.png}
        \caption{HepT1 test image}
      
    \end{subfigure}
  \begin{subfigure}{.5\textwidth}
        \centering
       \includegraphics[width=1\linewidth]{./figures/HepT1_seg_net2_98-13_20000_3.png}
        \caption{HepT1 predicted segmentation}
 \end{subfigure}

     \begin{subfigure}{.5\textwidth}
        \centering
        \includegraphics[width=1\linewidth]{./figures/Healthy_379-12b_20000_2.png}
        \caption{Healthy test image}
        \label{fig:healthy_label}
    \end{subfigure}
    \begin{subfigure}{.5\textwidth}
        \centering
        \includegraphics[width=1\linewidth]{./figures/Healthy_seg_net2_379-12b_20000_2.png}
        \caption{Healthy predicted segmentation}
        \label{fig:healthy_label}
    \end{subfigure}

\begin{subfigure}{.5\textwidth}
        \centering
        \includegraphics[width=1\linewidth]{./figures/Mhic1_825-14_200000_13.png}
        \caption{Mhic1 test image}
        \label{fig:healthy_label}
    \end{subfigure}
    \begin{subfigure}{.5\textwidth}
        \centering
        \includegraphics[width=1\linewidth]{./figures/Mhic1_seg_net2_827-14_20000_13.png}
        \caption{Mhic1 predicted segmentation}
        \label{fig:healthy_label}
    \end{subfigure}    

  \caption{Segmentation visualization of predicted images by the model trained with the SegUSK}
    \label{fig:seg_SegUSK} 
\end{figure}

\section{Mitochondrion-specific Structural Score}
With ClassUSK, we can calculate a normalized per-pixel score that measures the
structural transition from healthy to cancerous class by:
	 \begin{equation}
        \label{eq:cs_pixel}
            \Gamma_{pixel} = \frac {e^{s_2 - s_1}}{ e^{s_1 - s_2} + e^{s_2 - s_1} } 
   \end{equation}
where s1 and s2 are the probabilities to belong to the healthy mitochondria and 
cancerous mitochondria class, respectively. A weighted average of the per-pixel
scores within the segmented region of a single mitochondrion serves as a measure of
healthy-cancerous transition for the mitochondrial structure, hence:

   \begin{equation}
          \label{eq:cs_mitochondria}
              \Gamma_{mitochondria} =  \frac{\sum_{pixel \in mitochondrion} w_{pixel} \times \Gamma_{pixel}}{\sum_{pixel \in mitochondrion} w_{pixel}} , w_{pixel} = 1 - s_3
     \end{equation}
        \begin{equation}
          \label{eq:cs_mitochondria_2}
    \textrm {only if} \hspace{1em} \frac{1}{n_{pixel}} \sum \nolimits_{pixel \in mitochondrion} s_3 > 0.5
     \end{equation}
     
where the individual mitochondria segmentation is provided by the prediction
of SegUSK and $s_3$ is the probability belonging to non-mitochondrial class. We
dd the condition Eq. \ref{eq:cs_mitochondria_2} here to filter out the mistakes made by SegUSK in
segmenting impurities as mitochondria (Figure \ref{fig:schematic_seg_classification}).

\section{Acquisition of mitochondrial images}
Mitochondria from rat liver were isolated from cultured cells (McA7777, MH1C1,
H4IIE, Fao, HepG2 and HepT1) according to \cite{Schulz2015} . Electron microscopy
of the isolated mitochondria was done as described previously \cite{Schmitt2015}. Briefly, samples
were fixed in 2.5 $\%$ glutaraldehyde, postfixation and prestaining was done with
osmium tetroxide. After dehydration with ethanol and propylene oxide, samples were
embedded in Epon. Ultrathin sections were stained with uranylacetate and lead citrate
and examined with an EM 10 CR transmission electron microscope (Zeiss,
Germany). \\
