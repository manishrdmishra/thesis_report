\chapter{\abstractname}
	
	%TODO: Abstract
	
Mitochondria are the main source of cellular energy and thus
essential for cell survival. Pathological conditions like cancer, can cause
functional alterations and lead to mitochondrial dysfunction. Indeed,
electron micrographs of mitochondria that are isolated from cancer cells
show a different morphology as compared to mitochondria from healthy
cells. However, the description of mitochondrial morphology and the classification of the respective samples are so far qualitative. 
Furthermore, large intra-class variability and impurities such as mitochondrial fragments and other organelles in the micrographs make a clear separation
between healthy and cancerous samples challenging. In this study, we
propose a deep-learning based model to quantitatively assess the status
of each intact mitochondrion with a continuous score, which measures
its closeness to the healthy/tumor classes based on its morphology. 
This allows us to describe the structural transition from healthy to cancerous mitochondria. 
Methodologically, we train two USK networks, one to
segment individual mitochondria from an electron micrograph, and the
other to softly classify each image pixel as belonging to (i) healthy mitochondrial, (ii) cancerous mitochondrial and (iii) non-mitochondrial (image background \& impurities) tissue. Our combined model outperforms
each network alone in both pixel classification and object segmentation.
Moreover, our model can quantitatively assess the mitochondrial heterogeneity within and between healthy samples and different tumor types,
hence providing insightful information of mitochondrial alterations in
cancer development.

